

\renewcommand{\dateseparator}{-} %Omdefinierar hur datum visas

\renewcommand{\forening}{Revisorer}

\begin{center}
\LARGE{\textbf{Arbetsordning Revisorer}}
\end{center}
%\vspace{1cm}


\section{Uppdrag}
Revisorerna har i uppdrag av sektionsmötet att granska sektionens ekonomi och verksamhet.

\section{Allmänt}
\begin{itemize}
\item Revisor skall vara myndig.
\item Revisor skall ej inneha annat uppdrag på sektionen med ekonomiskt ansvar.
\item Revisorerna har närvaro- samt yttranderätt på styrelsemöte, studienämndsmöte samt kommittémöte.
\item Revisor skall i sitt arbete ej vara jävig.
\item Revisor skall ej granska sin direkta efterträdare.
\end{itemize}

\section{Generella åligganden}

\subsection{Det åligger revisorerna att:}
\begin{itemize}
\item på bästa sätt försäkra sig om att sektionens ekonomi sker enligt gällande bokföringslag.

\item granska att sektionens ekonomi sköts enligt de riktlinjer som bestämts av sektionsmötet och sektionsstyrelsen.

\item granska att sektionens verksamhet sker utefter stadga, reglemente samt övriga styrdokument och policyer.

\item kontinuerligt dokumentera sin verksamhet, som stöd till nästkommande års revisorer.

\end{itemize}

\subsection{Granskning av verksamhet}
\begin{itemize}
\item Revisorerna skall granska kallelser och protokoll till styrelsemöten.
\item Revisorerna skall granska kallelse, slutgiltig föredragningslista samt protokoll för sektionsmöte.
\end{itemize}


\subsection{Granskning av ekonomi}
\begin{itemize}
\item Revisorerna skall granska kvartalsrapporter för sektionsstyrelsen, studienämnden samt varje sektionskommitté efter varje kvartal, undantaget det kvartal då bokslut skall göras.

\item Revisorerna skall granska bokslut som inkommer under verksamhetsåret, undantaget bokslut som inkommer till första sektionsmötet som skall granskas av förra verksamhetsårets revisorer.

\item Kvartalsrapporter gällande verksamhetsårets sista kvartal granskas i samråd av avgående och tillträdande revisorer.

\item Revisorerna skall granska bokslut som inkommer i tillräcklig tid innan första sektionsmötet efter verksamhetsårets slut.

\end{itemize}

\subsubsection{Rapportering av ekonomi och verksamhet}
\begin{itemize}
\item Då tveksamheter upptäcks i sektionsstyrelsens ekonomi eller verksamhet skall detta rapporteras till sektionsmötet.

\item Då tveksamheter upptäcks i kommittéernas ekonomi eller verksamhet skall detta rapporteras till sektionsstyrelsen.

\item Då tveksamheter upptäcks i nämndens ekonomi eller verksamhet skall detta rapporteras till sektionsstyrelsen.

\item Då tveksamheter upptäcks i interesse- eller medlemsföreningars ekonomi eller verksamhet skall detta rapporteras till sektionsstyrelsen.

\end{itemize}

\section{Tolkning och ändring}
Tolkning av denna arbetsordning görs av sektionsstyrelsen.\\ Då Fysikteknologsektionens Stadga, Reglemente eller policyer motsäger denna arbetsordning har Stadga och Reglemente företräde. Då kårens Stadga, Reglemente eller policyer motsäger denna arbetsordning har kårens Stadga, Reglemente och policyer företräde.\\
Ändring och tillägg av denna arbetsordning görs av sektionsmöte.


Första versionen av denna arbetsordning antogs av sektionsmötet läsperiod 1 verksamhetsåret 14/15.

\newpage
