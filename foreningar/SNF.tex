\documentclass[a4paper]{article}
\usepackage{preamble}
\begin{document}

\renewcommand{\forening}{SNF} % Används för headern bl.a.

\begin{foreningenv}{\forening{}} % Hanterar mycket av logiken kring enskild/gemensam kompilering. Argumentet blir rubriken
    \section{Allmänt}
    \begin{itemize}
        \item SNF är sektionens studienämnd.
        \item SNF har nämndstatus på sektionen.
        \item SNF:s syfte är att sköta studiebevakningen på sektionen genom att kontinuerligt granska kurser som ges på programmen Teknisk fysik, Teknisk matematik samt associerade masterprogram.
    \end{itemize}
    
    SNF består av följande poster:
    \begin{itemize}
        \item Ordförande
        \item Vice ordförande
        \item Kassör
        \item Sekreterare
        \item Kandidatansvarig
        \item Masteransvarig
        \item Veckobladerist
        \item Årskursrepresentant åk. 1
        \item Matansvarig
    \end{itemize}
    Där ordförande, vice ordförande samt kassör anses vara förtroendeposter. \\
    
    Alla förtroendeposter nomineras av valberedningen, och övriga poster väljs av sektionsmötet enligt reglemente. Om någon av av övriga poster ej tillsätts enligt ovan kan de fyllnadsväljas av sektionsstyrelsen, men detta skall fastslås på påföljande sektionsmöte.
    
    \section{Generella åligganden}
    Det åligger SNF:
    \begin{description}
        \item[att] godkänna kursutvärderare.
        \item[att] sammanträda minst tre gånger per läsperiod.
        \item[att] informera F-och TM-teknologerna i frågor rörande respektive utbildning.
        \item[att] ansvara för utvecklandet av utbildningsbevakningen på Fysikteknologsektionen.
        \item[att] inför Fysikteknologsektionen svara för att F- och TM-teknologernas intressen i studiefrågor och studiemiljö bevakas på ett tillfredsställande sätt.
        \item[att] på verksamhetsårets första sektionsmöte presentera en verksamhetsplan för det kommande läsåret.
        \item[att] ordna arrangemang i studiebefrämjande syfte.
        \item[att] tillsammans med Fysikteknologsektionens valberedning lägga fram förslag på efterträdare till nämnden.
        \item[att] inför inval av poster och förtroendeposter i nämnden, tillse att samtliga sektionens medlemmar har givits fullgod möjlighet att med lätthet införskaffa information och bekantskap med nämnd och dess arbetsuppgifter.
        \item[att] representera Fysikteknologsektionen i F:s och TM:s programråd.
        \item[att] delta i de av Djungelpatrullen anordnade städdagarna två gånger per läsår. Om detta uppfylls får de närvarande gå gratis på nästa sektionsaktivafest.
        \item[att] under inga omständigheter vid med sektionen associerade arrangemang eller med sektionen associera festlighet servera, försälja eller bjuda på öhl av märket Pripps, förutom vid häfv eller arrangemang i kårhuset.
    \end{description}
    
    \subsection{Övrigt}
    Därtill åligger det SNF:
    \begin{description}
        \item[att] i samband med Chalmers studentkårs pubrunda anordna Cocktailparty i varje läsperiod.
    \end{description}
    
    \subsection{Valberedning}
    \aliggvalber{}
    
    \section{Postspecifika uppdrag}
    \subsection{Ordförande}
    SNF:s ordförande är tillsammans med SNF:s kassör ansvarig för kommitténs ekonomi. Ordförande skall leda SNF:s arbete samt vara en kontaktlänk mellan SNF och övriga kommittéer, nämnder samt sektionsstyrelsen. Det åligger SNF:s ordförande att tillse att nämndens åligganden utförs.\\
    
    \begin{itemize}
        \item SNF:s ordförande är ledamot i sektionsstyrelsen.
    \end{itemize}
    
    Det åligger SNF:s ordförande:
    \begin{description}
        \item[att] tillse att studienämndens åligganden utförs. 
        \item[att] leda studienämndens verksamhet.
        \item[att] kalla studienämnden till sammanträde.
        \item[att] handha studienämndens handlingar.
        \item[att] underteckna studienämndens handlingar.
        \item[att] i studie- och studiemiljöfrågor representera Fysikteknologsektionen och föra dess talan.
        \item[att] representera F och TM i Utbildningsutskottet, UU, och vid förhinder tillse att suppleant deltager.
    \end{description}
        
    \subsection{Vice ordförande}
    Det åligger vice ordföranden:
    \begin{description}
        \item[att] assistera ordföranden i dennes åligganden.
        \item[att] ersätta ordföranden när denne inte är närvarande. 
        \item[att] vara studienämndens suppleant i sektionsstyrelsen.
        \item[att] ansvara för studiesociala evenemang.
        \item[att] ansvara för av SNF arrangerade räkneövningstillfällen.
    \end{description}
    
    \subsection{Kassör}
    Kassören är tillsammans med ordförande ansvarig för SNF:s ekonomi. Det åligger kassören att kontinuerligt föra en granskningsbar redovisning gällande SNF:s ekonomi.\\
    
    Det åligger kassören:
    \begin{description}
        \item[att] mot revisorerna kontinuerligt redovisa för den ekonomiska situationen.
        \item[att] sköta och ansvara för studienämndens ekonomi tillsammans med ordförande.
        \item[att] kontinuerligt föra en granskningsbar redovisning gällande studienämdens ekonomi.
    \end{description}
    
    \subsection{Kandidatansvarig}
    Det åligger SNF:s kandidatansvarig:
    \begin{description}
          \item[att] ansvara för att kursutvärderingsansvariga tillsätts.
    \end{description}
    
    \subsection{Veckobladerist}
    Det åligger Veckobladeristen (VBL):
    \begin{description}
          \item[att] hålla Veckobladeriets hemsida uppdaterad.
          \item[att] komplettera Veckobladeriets arkiv så att de är aktuella.
          \item[att] vårda minnet av det enorma arbetet de gamla Veckobladeristerna har genomfört genom att en gång per år anordna en omsits åt dem. 
    \end{description}
    
    \subsection{Sekreterare}
    Det åligger SNF:s sekreterare:
    \begin{description}
        \item[att] tillse att protokoll förs på studienämndens möten.
        \item[att] anslå nämndens protokoll senast efter 2 läsveckor via sektionens officiella kommunikationskanaler. 
    \end{description}
    
    \subsection{SNF:s årskursrepresentant}
        Det åligger SNF:s årskursrepresentant:
    \begin{description}
        \item[att] speciellt bevaka studiefrågor och studiesociala miljö i första årskursen.
        \item[att] i första årskursen informera om studienämndens verksamhet.
        \item[att] anordna övningstentor för förstaårsstudenterna på F och TM inför första tentorna.
    \end{description}
    
    \subsection{Mastersansvarig}
    Det åligger SNF:s mastersansvarig:
    \begin{description}
          \item [att] handha studiebevakningen på masterprogram associerade med programmet Teknisk fysik och/eller programmet Teknisk matematik.
          \item [att] representera studienämnden på masterprogram associerade med
          programmet Teknisk fysik och/eller programmet Teknisk matematik.
          \item [att] representera studienämnden i Fysikteknologsektionens Masterenhet.
    \end{description}
        
    \subsection{SNF :s medlemmar}
    SNF:s medlemmar och ledamöter skall hjälpa SNF:s ordförande i dennes uppgifter så att de utförs på bästa sätt.
    
    Det åligger SNF:s medlemmar:
    \begin{description}
        \item[att] hjälpa VBL att komplettera Veckobladeriets arkiv.
        \item[att] i frånvaro av VBL utföra VBL:s uppgifter.
        \item[att] vara ordföranden behjälplig.
    \end{description}
\end{foreningenv}

\end{document}