% Inaktuell; Styrets SAMO är numera ensam SAMO.

\renewcommand{\dateseparator}{-} %Omdefinierar hur datum visas

\renewcommand{\forening}{SAMO}

\begin{center}
\LARGE{\textbf{Arbetsordning \forening}}
\end{center}
%\vspace{1cm}



\section{Uppdrag}
Sektionens Studerandearbetsmiljöombud, SAMO, har i uppdrag att granska den fysiska och psykosociala arbetsmiljön som studenter på fysikteknologsektionen utsätts för.
Fysikteknologsektionen har tre SAMO:s, två stycken oberoende SAMO utsedda av sektionsmötet från programmen Teknisk Fysik respektive Teknisk Matematik samt sektionstyrelsens skyddsombud.

\section{Allmänt}
\begin{itemize}
\item SAMO:s arbete leds av sektionsstyrelsens skyddsombud.
\item Oberoende SAMO får ej inneha post i sektionsstyrelsen, studienämnden eller någon sektionskommitté.

\item SAMO har i sitt uppdrag tystnadsplikt.

\item SAMO har närvaro- samt yttranderätt på styrelsemöte, studienämndsmöte samt kommittémöte.

\end{itemize}

\section{Granskning av fysisk arbetsmiljö}
\begin{itemize}
\item SAMO skall deltaga i de arrangerade skyddsronder samt uppföljningsmöte gällande fysisk arbetsmiljö.

\item SAMO skall ta emot klagomål gällande fysiska arbetsmiljö från sektionens medlemmar.

\end{itemize}


\section{Granskning av psykosocial arbetsmiljö}
\begin{itemize}
\item SAMO skall deltaga i de arrangerade skyddsronder samt uppföljningsmöte gällande psykosocial arbetsmiljö.

\item SAMO skall ta emot klagomål gällande psykosocial arbetsmiljö från sektionens medlemmar.
\end{itemize}

\section{Rapportering av verksamhet}
\begin{itemize}
\item Då SAMO ombeds att behandla problem gällande fysisk eller psykosocial arbetsmiljö måste detta behandlas.

\item Då problem eller potentiella problem med den fysiska eller psykosociala arbetsmiljön upptäcks ska detta rapporteras till lämpligt organ.

\item SAMO skall i första hand rapportera problem gällande fysisk och psykosocial arbetsmiljö till sektionsstyrelsen.

\item Då oberoende SAMO får vetskap om problem gällande fysisk eller psykosocial arbetsmiljö där sektionsstyrelsen ej anses vara lämpligt att rapportera detta, skall detta rapporteras till högre organ så som sociala enheten i kåren, kårstyrelsen eller kurator.

\end{itemize}

\section{Tolkning och ändring}
Tolkning av denna arbetsordning görs av sektionsstyrelsen.\\ Då Fysikteknologsektionens Stadga, Reglemente eller policyer motsäger denna arbetsordning har Stadga och Reglemente företräde. Då kårens Stadga, Reglemente eller policyer motsäger denna arbetsordning har kårens Stadga, Reglemente och policyer företräde.\\
Ändring och tillägg av denna arbetsordning görs av sektionsmöte.


Första versionen av denna arbetsordning antogs av sektionsmötet läsperiod 1 verksamhetsåret 14/15.

\newpage