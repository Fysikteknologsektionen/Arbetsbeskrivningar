\documentclass[a4paper]{article}
\usepackage{preamble}
\begin{document}

\renewcommand{\forening}{Game Boy} % Används för headern bl.a.

\begin{foreningenv}{\forening{}} % Hanterar mycket av logiken kring enskild/gemensam kompilering. Argumentet blir rubriken
    \section{Allmänt}
    \begin{itemize}
        \item Game Boy består av 0--6 Game Boys.
        \item Game Boy har funktionärsstatus på sektionen.
        \item Game Boys syfte är att ta hand om de sällskapsspel som finns på Focus, samt främja brädspelsverksamheten på sektionen
    \end{itemize}
    Game Boy består av 0--6 Game Boys.
    
    \section{Åligganden}
    \aliggsektfunkt{}
    
    \subsection{Det åligger Game Boy:}
    \begin{description}
        \item[att] komplettera spelen på Focus så att de är spelbara och så att regler till alla spelen finns.
        \item[att] ansvara för införskaffande av nya spel enligt medel anslagna av sektionsstyrelsen.
        \item[att] bidra till att förenkla för F-teknologen att som ny spelare förstå hur spelet fungerar.
        \item[att] som oberoende part agera domare i frågor angående hur spelregler ska tolkas. Gäller inte då personen i fråga själv är med i spelet då denne i det fallet anses vara jävig.
        \item[att] främja brädspelsverksamheten på Fysikteknologsektionen.
    \end{description}
\end{foreningenv}

\end{document}