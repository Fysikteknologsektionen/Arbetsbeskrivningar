\documentclass[a4paper]{article}
\usepackage{preamble}
\begin{document}

\renewcommand{\forening}{Focumateriet} % Används för headern bl.a.

\begin{foreningenv}{\forening{}} % Hanterar mycket av logiken kring enskild/gemensam kompilering. Argumentet blir rubriken
    \section{Allmänt}
    \begin{itemize}
        \item Focumateriet är sektionens Focumateri.
        \item Focumateriet har kommittéstatus på sektionen.
        \item Focumateriets syfte är att sköta automater och flipperspel som ägs av sektionen och befinner sig i sektionslokalen eller i direkt närhet till sektionslokalen.
    \end{itemize}
    
    Focumateriet består av följande poster:
    \begin{itemize}
        \item Ordförande, Kaptenen
        \item Vice ordförande, Automatpirat
        \item Kassör, Kistväktaren
        \item 0--5 övriga ledamöter
    \end{itemize}
    Där Ordförande, vice ordförande och kassör är förtroendeposter.
    
    De övriga ledamöterna utses i nära anslutning till val av förtroendeposter och tjänstgör under ett års tid. De ska genomgå ett samtal med valberedningen innan de kan rekommenderas för tjänstgöring. Övriga ledamöter skall godkännas av sektionsstyrelsen. Om någon av de övriga ledamöterna ej tillsätts enligt ovan kan de fyllnadsväljas av sektionsstyrelsen i samråd med sittande förtroendevalda.
    
    \section{Generella åligganden}
    \aliggkom{}
    
    \section{Specifika åligganden:}
    Det åligger Focumateriet:
    \begin{description}
        \item[att] handha Focumaten, samt, efter sektionsstyrelsens bestämmande, av sektionen ägda automater samt av sektionen ägd elektronisk utrustning.
        \item[att] vid sektionstillställningar, eller vid sektionstillställning jämförbar fest, bära haklapp och spypåse samt ha städdon i omedelbar närhet för att få närvara. F6 bedömmer enväldigt och utan chans till överklagan huruvida förutsättningarna enligt ovan är uppfyllda.
        \item[att] se till att teknologerna har tillgång till kaffe, te och socker i sektionslokalen till självkostnadspris.
        \item[att] inför sektionsmötet LP4, med styrets godkännande, bestämma den hatt som nästkommande talmanspresidium ska använda.
    \end{description}
    
    \subsection{Valberedning}
    \aliggvalber{}
    
    \section{Postspecifika uppdrag}
    \subsection{Ordförande, Kaptenen}
    Focumateriets ordförande är tillsammans med Focumateriets kassör ansvarig för kommitténs ekonomi. Ordförande skall leda Focumateriets arbete samt vara en kontaktlänk mellan Focumateriet och övriga kommittéer, nämnder samt sektionsstyrelsen. 
    
    \begin{itemize}
        \item Focumateriets ordförande är ledamot i sektionsstyrelsen.
    \end{itemize}
    
    Det åligger Focumateriets ordförande:
    \begin{description}
        \item[att] tillse att kommitténs åligganden utförs.
        \item[att] leda Focumateriets arbete. 
        \item[att] i kassörens frånvaro sköta Focumateriets ekonomi. 
    \end{description}
    
    
    \subsection{Vice ordförande, Automatpiraten}
    Det åligger Focumateriets vice ordförande, Styrmannen:
    \begin{description}
        \item[att] vara kommitténs suppleant i sektionsstyrelsen.
        \item[att] bistå ordföranden i dennes uppgifter så att de utförs på bästa sätt.
        \item[att] i ordförandens frånvaro utföra dennes uppdrag.
        \item[att] ansvara för sektionens automater.
    \end{description}
    
    \subsection{Kassör}
        Kassören är tillsammans med ordföranden ansvarig för Focumateriets ekonomi.
    
    \subsubsection{Det åligger kassören i varje sektionskommitté:}
    \begin{description}
        \item[att] kontinuerligt föra en granskningsbar redovisning gällande kommitténs ekonomi.
        \item[att] mot revisorerna kontinuerligt redovisa för den ekonomiska situationen.
    \end{description}
    
    
    \subsection{Övriga ledamöter}
    Focumateriets övriga ledamöter skall hjälpa Focumateriet s ordförande i dennes uppgifter så att de utförs på bästa sätt.
\end{foreningenv}

\end{document}