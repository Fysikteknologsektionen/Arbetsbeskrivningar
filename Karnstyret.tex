

\renewcommand{\dateseparator}{-} %Omdefinierar hur datum visas

\renewcommand{\forening}{Kärstyret }

\begin{center}
\LARGE{\textbf{Arbetsordning Kärnstyret}}
\end{center}
%\vspace{1cm}



\section{Allmänt}
Kärnstyret är ett samlingsnamn för de medlemmar av sektionsstyrelsen som ej är ordföranden av kommitté eller nämnd. Kärnstyret innefattar därmed sektionsstyrelsens ordförande, vice ordförande, kassör, sekreterare, skyddsombud samt informationsansvarig.

\section{Generella åligganden}
Kärnstyret har till uppgift att tillse att praktiska uppgifter mellan styrelsemöten verkställs. Kärnstyret skall tillse att övriga medlemmar av sektionsstyrelsen får bästa tänkbara underlag för sina beslut.

\subsection{Övrigt:}

\subsubsection{Valberedning}
Inför valberedningen är kärnstyret skyldig att bilda sig en detaljerad uppfattning om varje aspirant till kärnstyret som skall valberedas.\\
Därtill är kärnstyret skyldig att i samråd med valberedningen upprätta en kravprofil för kärnstyret i helhet samt för de enskilda posterna.



\section{Postspecifika åligganden}

\subsection{Ordförande}
Sektionens ordförande tecknar sektionens firma och är tillsammans med kassören ansvarig för sektionens ekonomi.\\
Ordföranden ska leda och övervaka sektionsstyrelsens arbeta och har det yttersta ansvaret för att sektionens beslut verkställs.\\
Sektionsordförande har full insyn i F-teknologsektionens alla organ och äger rätt att deltaga i deras möten med yttranderätt.\\
Därtill åligger det sektionens ordförande:

\begin{description}
\item[att] tillse att sektionens beslut verkställs. 
\item[att] föra sektionens talan då något annat ej stadgats eller beslutats
\item[att] sammankalla sektionen till sektionsmöten
\item[att] vara Fysikteknologsektionens representant i kårledningsutskottet
\item[att] tillse att det finns representanter från sektionsstyrelsen i F och TM:s programråd.
\item[att] vid sektionsmöte bära skägg som kliar.
\item[att] vara innehavare av sektionens serveringstillstånd.
\end{description}

\subsection{Vice ordförande}
Sektionens vice ordförande skall bistå ordföranden i dess arbete och tar vid dess frånvaro över ordförandens befogenheter och åligganden. \\
Därtill åligger det sektionens vice ordförande:
\begin{description}
\item[att] i ordförandes frånvaro överta dennes åligganden. 
\item[att] tillse att sektionens kommittéer sköter sina åligganden. 
\item[att] i samråd med styrelsen och övriga funktionärer upprätta sektionens verksamhetsberättelse.
\item[att] två till fem gånger per läsår kalla samtliga komittéer, studienämden samt berörda föreningar och funktionärer till ett stormöte. 
\item[att] vara Fysikteknologsektionens representant i kårens nöjeslivsutskott. 
\item[att] ansvara för att det finns en första hjälpen-låda i Focus bardel med lämpligt innehåll. 
\end{description}

\subsection{Kassör}
Sektionens kassör tecknar sektionens firma och är tillsammans med ordföranden ansvarig för sektionens ekonomi. \\
Därtill åligger sektionskassören:
\begin{description}
\item[att] sköta och ansvara för Fysikteknologsektionens ekonomi tillsammans med ordföranden.
      \item[att] fortlöpande kontrollera kommittéernas räkenskaper och bokför\-ing.
      \item[att] teckna sektionens firma.
      \item[att] genom Chalmers Studentkår uppbära sektionsavgiften.
      \item[att] i samråd med sektionsstyrelsen upprätta preliminärt budgetförslag till första ordinarie höstmötet.
      \item[att] till varje sektionsmöte kunna redogöra för sektionens ekonomiska ställning.
      \item[att] informera nya ekonomiskt ansvariga om sektionens bokförings- och redovisningssystem.
      \item[att] sammankalla till kassörsmöte minst en gång per läsperiod.
      \item[att] ansvara för försäljning av tröjor, märken och dylikt.
      \item[att] vara Fysikteknologsektionens representant i kårens ekonomiforum. 
\end{description}

\subsection{Sekreterare}
Det åligger sektionens sekreterare:
\begin{description}

\item[att] föra protokoll vid styrelsemöten och senast två läsdagar efter möte
överräcka renskrivet protokoll till ordföranden.
\item[att] tillse att protokoll från såväl styrelse- som sektionsmöten anslås.
\item[att] tillse att sektionens stadgar, reglemente och förordningar är aktuella och efterlevs.
\item[att] handha överlämning av nycklar till de olika föreningsrummen och
föra förteckning över dem.



\end{description}
\subsection{Skyddsombud}
Sektionensstyrelsens skyddsombud är studerandearbetsmiljöombud, SAMO, tillika jämlikhetsansvarig på sektionen.\\
Därtill åligger det sektionsstyrelsens skyddsombud:
\begin{description}

\item[att] i enlighet med reglementet punkt 10.7 vara ett av sektionens studerandearbetsmiljöombud, samt vara sektionens jämlikhetsansvarige.
\item[att] vara Fysikteknologsektionens representant i kårens sociala utskott. 
\end{description}

\subsection{Informationsansvarig}
Sektionsstyrelsens Informationsansvarige ansvarar för sektionens kommunikationskanaler.\\ Därtill åligger det sektionsstyrelsens Informationsansvarige:
\begin{description}

\item[att] tillse att material som inkommer till sektionen anslås eller på annat sätt förmedlas till den/dem de berör
\item[att] handha Fysikteknologsektionens korrespondens.
\item[att] tillse att informationen på Fysikteknologsektionens internetportal
uppdateras kontinuerligt.
\item[att] vara Fysikteknologsektionens representant i kårens informationsutskott.
\item[att] Ansvara för sektionens närvaro på sociala medier.
\end{description}

\section{Tolkning och ändring}
Tolkning av denna arbetsordning görs av sektionsstyrelsen.\\ Då Fysikteknologsektionens Stadga, Reglemente eller policyer motsäger denna arbetsordning har Stadga och Reglemente företräde. Då kårens Stadga, Reglemente eller policyer motsäger denna arbetsordning har kårens Stadga, Reglemente och policyer företräde.\\
Ändring och tillägg av denna arbetsordning görs av sektionsstyrelsen. Ändring skall redovisas vid nästkommande sektionsmöte. 
\newpage