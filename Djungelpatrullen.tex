

\renewcommand{\dateseparator}{-} %Omdefinierar hur datum visas

\renewcommand{\forening}{Djungelpatrullen}

\begin{center}
\LARGE{\textbf{Arbetsordning \forening}}
\end{center}
%\vspace{1cm}


\section{Allmänt}
\begin{itemize}
\item \forening \ är sektionens rustmästeri och PR-förening.
\item \forening \ har kommittéstatus på sektionen.
\item \forening s syfte är att tillse att sektionshelgonet vördas på ett hederssamt sätt av alla medlemmar. \forening \ ansvarar dessutom för det löpande underhållet av sektionslokalen samt vårdar sektionens traditioner
\end{itemize}

\forening  består av följande poster:
\begin{itemize}
\item Ordförande, Översten
\item Vice ordförande, Rustmästaren
\item Kassör, Skattmästaren
%\item Öhlchef
\item 0-7 adjutanter
\end{itemize}

Där ordförande, Vice ordförande, och Kassör  är förtroendeposter. \\

Adjutanterna utses i nära anslutning till val av förtroendeposter och tjänstgör under ett års tid. 
De ska genomgå ett samtal med valberedningen innan de kan rekommenderas för tjänstgöring. Adjutanterna skall godkännas av sektionsstyrelsen. 
Om någon av de övriga medlemmarna ej tillsätts enligt ovan kan de fyllnadsväljas av sektionsstyrelsen i samråd med sittande förtroendevalda.

\section{Generella åligganden}

Det åligger varje sektionskommitté:
\begin{description}
    \item[att] på det ordinarie sektionsmöte som följer på det, då kommittén blivit invald, presentera en verksamhetsplan för det kommande verksamhetsåret.
      \item[att] tillsammans med Fysikteknologsektionens valberedning lägga fram förslag på efterträdare till respektive sektionskommitté.
      \item[att] inför inval av poster och förtroendeposter i respektive sektionskommitté, tillse att samtliga sektionens medlemmar har givits fullgod möjlighet att med lätthet införskaffa information och bekantskap med respektive sektionskommitté och dess arbetsuppgifter.
      \item[att] kontinuerligt dokumentera sin verksamhet, som stöd till näst\-komm\-ande års kommitté.
      \item[att] delta i de av Djungelpatrullen anordnade städdagarna två gånger per
      läsår. Om detta uppfylls får de närvarande gå gratis på nästa
      sektionsaktivafest.
      \item[att] under inga omständigheter vid med sektionen associerat arrangemang eller vid med sektionen associerad festlighet servera, försälja eller bjuda på öhl av märket Pripps, förutom vid häfv eller arrangemang i kårhuset.
      
    \end{description}



\section{Specifika åligganden}
Det åligger \forening:
\subsection {Allmänt}
\begin{description}
\item[att] ansvara för att det ordnas arrangemang för sektionens medlemmar av sådant slag att sammanhållningen och kontakten över årskursgränserna på sektionen främjas.

\item[att] hålla i Fysikteknologsektionens öhlhäfv.

\item[att] vara ett komplement till FnollK under mottagningen

\item[att] vårda sektionens traditioner.

\item[att] ansvara för det löpande underhållet av sektionens lokaler och egendom. 

\end{description}


\subsection{Medlemmar och representanter}

\begin{description}
\item[att] Inför nollbalen utse en representant bland sina medlemmar att vara Balnågonting behjälplig inför och under arrangerandet av balen.
%\item[att] \sout{Bland sina adjutanter välja en att inneha posten överflödig. }
\end{description}

\subsection{Pubrundan}
\begin{description}
\item[att] Ansvara för att sektionen deltar med en pub under Chalmers studentkårs pubrunda varje läsperiod.

\end{description}


\subsection{Övrigt}
\begin{description}
\item[att] I samråd med FnollK och F6 tillse att sektionsmärket i Olgas trappor målas i samband med mottagningen.

\item[att] Varje onsdag läsvecka 3 servera sektionens medlemmar hofflor på sektionens bekostnad.

\item[att] En gång per termin arrangera en städdag där samtliga sektionskommittéer, nämnder, sektionsstyrelsen samt sektionsfunktionärer deltar och grundligt städar sektionslokalen med omnejd.

\item[att] Ansvara för att Fysikteknologsektionen prenumererar på tidning-
en Fantomen samt ordna så att tidningen finns tillgänglig för
sektionsmedlemmarna.

\item[att] Tillhandahålla instruktioner för tillverkning av sektionsoverall.
\end{description}

\subsection{Valberedning}
\subsubsection{Förberedelse}
Inför valberedningen är kommittén skyldig att bilda sig en detaljerad uppfattning om varje aspirant till föreningen som skall valberedas.\\
Därtill och kommittén skyldig att i samråd med valberedningen upprätta en kravprofil för kommittén i helhet samt för kommitténs förtroendeposter.

\subsubsection{Intervjuer}
Under valberedningens intervjuer skall \forening \ utse två representanter att deltaga under dessa samt under de efterföljande samtalen i syfte att nominera kandidater.\\
De representanter som utses får själv inte valberedas för eller söka någon post i denna förening.

\section{Postspecifika uppdrag}

\subsection{Ordförande, Översten}
\subsection{Allmänt}
\forenings ordförande är tillsammans med \forening s kassör ansvarig för kommitténs ekonomi. Ordförande skall leda \forening s arbete samt vara en kontaktlänk mellan \forening \ och övriga kommittéer, nämnder samt sektionsstyrelsen. 
\begin{itemize}
\item \forening s ordförande är ledamot i sektionsstyrelsen.

\end{itemize}

\subsubsection{Åligganden}
Det åligger Djungelpatrullens Ordförande:
\begin{description}
\item[att] tillse att kommitténs åligganden utförs.

\item[att] vid med sektionen associerat arrangemang eller med sektionen associerad festlighet, för sin egen säkerhet bära suspensoar. 

\end{description}

\subsection{Vice ordförande, Rustmästaren}
Det åligger vice ordföranden:
\begin{description}
\item[att] I ordförandens frånvara utföra dennes uppdrag.

\item[att] organisera sektionsmedlemmarnas städtjänstgöring.

\item[att] leda renoverings- och underhållsarbeten i sektionens lokaler.

\item[att] ansvara för uthyrning av sektionslokalens inventarier.

\item[att] vara kommitténs suppleant i sektionsstyrelsen

\end{description}


\subsection{Kassör, Skattmästaren}
Kassören är tillsammans med ordföranden ansvarig för \forening  ekonomi. Det åligger kassören att kontinuerligt föra en granskningsbar redovisning gällande \forening s ekonomi.\\

\subsubsection{Åligganden}
Det åligger kassören i varje sektionskommitté:
\begin{description}
\item[att] mot revisorerna kontinuerligt redovisa för den ekonomiska situationen.

\end{description}

\subsection{Öhlchef}
Det åligger \forening \ att bland sina adjutanter välja en öhlchef. \\
Det åligger \forening s öhlchef:
\begin{description}
\item[att] ansvara för inköp samt hantering av kommitténs alkohol under arrangemang.
\item[att] vara serveringsansvarig.
\end{description}


\subsection{Adjutanter}
\forening s adjutanter skall vara de förtroendevalda i \forening  behjälpliga i verksamheten. Dessa skall dessutom utefter bästa förmåga hjälpa översten att upprätthålla en god stämning, både inom kommittén och på sina arrangemang.

\subsubsection{Överflödig }
Adjutant som innehar posten ''Överflödig'' skall när helst de bär overall, och då lagen så tillåter, alltid bära med sig kall Skånes Aquavit samt glas.

\section{Tolkning och ändring}
Tolkning av denna arbetsordning görs av sektionsstyrelsen.\\ Då Fysikteknologsektionens Stadga, Reglemente eller policyer motsäger denna arbetsordning har Stadga och Reglemente företräde. Då kårens Stadga, Reglemente eller policyer motsäger denna arbetsordning har kårens Stadga, Reglemente och policyer företräde.\\
Ändring och tillägg av denna arbetsordning görs av sektionsstyrelsen. Ändring skall redovisas vid nästkommande sektionsmöte. 


Första versionen av denna arbetsordning antogs av sektionsmötet läsperiod 1 verksamhetsåret 14/15.

\newpage
